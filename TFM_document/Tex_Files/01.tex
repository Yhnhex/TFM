\chapter{Introduction} \label{ch:chapter_01}
In the context of space exploration, the thermal management of spacecraft is a critical aspect, particularly in ensuring their resilience to extreme temperature conditions. This study is dedicated to refining the methodologies employed in temperature sensor instrumentation for vacuum thermal testing and for on-board measurements, with a specific emphasis on the UPMSat-3 satellite.

The primary objective is to establish a systematic approach that enhances our ability to extract comprehensive thermal behavior information from spacecraft components during testing. This involves the development of Python and C++ code for the processing and analysis of thermal models from ESATAN. These models encompass both the flight and test configurations of UPMSat-3, facilitating a detailed examination of its thermal characteristics.

Additionally, the study entails the computation of influence and sensitivity matrices, elucidating the relationships between temperature and various parameters. The identification of independent thermal parameters is integral to characterizing the spacecraft's thermal response accurately. Furthermore, a method for strategically positioning temperature sensors during tests will be devised to efficiently determine the values of these identified thermal parameters.

The culmination of these efforts will be the application of the developed methodology to the thermal testing of UPMSat-3. This case study serves not only to validate the proposed strategies but also to contribute valuable insights to the broader field of space vehicle thermal design.

Through the interplay of theoretical model processing, parameter identification, and strategic sensor placement, this research aims to elevate the precision and efficiency of thermal testing. Subsequent sections will provide detailed insights into the specific methodologies employed to achieve the outlined objectives.

\section{Spacecraft Thermal Control}

The thermal design of a spacecraft is primarily influenced by the conditions it encounters during its mission in space. From a thermal perspective, the space environment is characterized by a vacuum, Solar radiation (both direct and reflected by nearby planets, known as albedo), and the infrared emission of celestial bodies.

The main driver of a spacecraft thermal design is the in-flight environment where it needs to operate. From a thermal point of view, the space environment is characterized by the vacuum, the incoming Solar radiation, both direct and reflected by a nearby planet (albedo), and the infrared
emission of the planet.Because a spacecraft operates in a vacuum, the only possible thermal interaction between the spacecraft and its environment its through radiation. On an Earth orbit, solar irradiance is the main heat load, with a mean value of 1366 W/m2 and a seasonal variation of ±1.7\% due to the eccentricity of the orbit of the Earth around the Sun. The solar irradiance value scales with the square of the distance to the Sun, and its spectrum can be modeled, from a thermal point of view, as a black body at some 5762 K, where 99 \% of the spectral emissive power of the Sun lies in the range 0.15 to 10 µm wavelength.

\subsection{Thermal mathematical modelling}
\subsection{Analaysis cases}
\subsection{Reduced thermal mathematical models}

\chapter{Mathematical formulation}
\section{Problem definition}
\section{Error definition}
\section{Model requirements}
\section{Data acquisition}
\section{Observability}
\section{Parameters and nodes reduction}
In order to choose the most adequate parameters to determine the reduced model, the matrix of influence $\mathbf{I}_{\mathbf{X}}$ is defined below:
\begin{equation}
\mathbf{I}_{\mathbf{X}}=\left[\begin{array}{cccc}
\frac{\partial T_1}{\partial X_1} \delta X_1 & \frac{\partial T_1}{\partial X_2} \delta X_2 & \ldots & \frac{\partial T_1}{\partial X_{N_P}} \delta X_{N_P} \\
\ldots & \ldots & & \ldots \\
\frac{\partial T_{N_N}}{\partial X_1} \delta X_1 & \frac{\partial T_{N_N}}{\partial X_2} \delta X_2 & \ldots & \frac{\partial T_{N_N}}{\partial X_{N_P}} \delta X_{N_P}
\end{array}\right]=\mathbf{M} \delta \boldsymbol{X}
\end{equation}
where $\mathbf{M}$ is the jacobian or sensibility matrix and $ \delta \boldsymbol{X}$ is a vector containing the allowable variation of each parameter within the design. In the influence matrix $\mathbf{I}_{\mathbf{X}}$ each column represents the temperature variation of the nodes that would be generated by a deviation on the parameter $ \delta X_i$. Therefore, the elements of this matrix have dimensions of temperature, showing the effect of every parameter in the model, which would not be possible using the jacobian matrix directly
\section{Sensor positioning} 


\chapter{Application to a 4 nodes models}
\chapter{Application to the UPMSat-3}
\section{Context}
\section{Thermal mathematical model}
\section{Model reduction}
\subsection{Parameter identification}
\begin{figure}[H]
	\centering
	\includegraphics[width=\textwidth]{Figures/figs_malas/infmatHot_redAverage.png}
	\caption{NO SIRVE, ES V0.}
	\label{fig:fm1}
\end{figure}
\begin{figure}[H]
	\centering
	\includegraphics[width=\textwidth]{Figures/figs_malas/infmatCold_redAverage.png}
	\caption{NO SIRVE, ES V0}
	\label{fig:fm2}
\end{figure}
\begin{figure}[H]
	\centering
	\includegraphics[width=\textwidth]{Figures/figs_malas/infGlobalHot.png}
	\caption{NO SIRVE, ES V0}
	\label{fig:fm3}
\end{figure}
\begin{figure}[H]
	\centering
	\includegraphics[width=\textwidth]{Figures/figs_malas/infGlobalCold.png}
	\caption{NO SIRVE, ES V0}
	\label{fig:fm4}
\end{figure}
\subsection{Nodal reduction}
\subsection{Results}



% \section{Sección}

% \subsection{Citar referencias y acrónimos}

% \cite{im78552}, \acrshort{mc}, \acrfull{mc}.

% \subsection{Enumeraciones}

% Enumeración.

% \begin{enumerate}
% 	\item La impresora debe contar con un sistema de nivelación de la base de impresión.
% 	\item El sistema de extrusión de filamento debe asegurar que no se producirán inconsistencias durante los periodos largos de trabajo.
% \end{enumerate}

% Enumeración cambiando los items.

% \begin{enumerate}[label= \textbf{R-\arabic*}]
% 	\item La impresora debe contar con un sistema de nivelación de la base de impresión.
% 	\item El sistema de extrusión de filamento debe asegurar que no se producirán inconsistencias durante los periodos largos de trabajo. \label{req:extrusion}
% \end{enumerate}

% Referenciar un item: \ref{req:extrusion}, \autoref{req:extrusion}.


% Ejemplo de bulletpoints.


% \begin{itemize}[label={\scriptsize\raisebox{0.5ex}{\textbullet}}]

% 	\item Perfilería de aluminio.

% \end{itemize}



% %   ---   ---   %

% \subsection{Figuras}

% Las figuras se pueden fijar en el texto con H, posicionarlas lo mejor posible con h!, arriba con t, etc.

% \begin{figure}[H]
% 	\centering
% 	\includegraphics[width=100mm]{duet3}
% 	\caption{Motherboard Duet 3 6HC.}
% 	\label{fig:000_00}
% \end{figure}

% Subfiguras en paralelo.

% \begin{figure}[H]
% 	\centering
% 	\subfloat[Vista frontal del montaje.\label{fig:mont1}]
% 	{
% 		\includegraphics[width=50mm,angle=0]{Mont_01}
% 	}
% 	\hspace*{10mm}
% 	\subfloat[Vista trasera del montaje.\label{fig:mont2}]
% 	{\includegraphics[width=50mm,angle=0]{Mont_02}
% 	}
% 	\caption{Montaje del sistema de transmisión del eje X}
% 	\label{fig:mont_nema}
% \end{figure}


% Ejemplo dos figuras en paralelo centradas verticalemtne.

% \begin{figure}[H]
% 	\begin{minipage}{\textwidth}
% 		\centering
% 		\raisebox{-0.5\height}{\includegraphics[width=0.4\textwidth]{Perfil_Aluminio}}
% 	\hspace*{.2in}
% 		\raisebox{-0.5\height}{\includegraphics[width=0.25\textwidth]{Perfil_Aluminio_Union}
% 		}
% 	\end{minipage}
% 	\caption{Ejemplos de perfilería de aluminio.}
% 	\label{fig:perfileria_alumnio}
% \end{figure}

% Barrera que no pueden atravesar las figuras.
% \FloatBarrier


% %   ---   ---   %

% \subsection{Ecuaciones}

% Ejemplo de ecuación, \autoref{eq:velocidad}

% \begin{equation}\label{eq:velocidad}
% 	l = \frac{\pi\cdot1.8}{180\cdot P}\cdot r = \frac{\pi\cdot1.8}{180\cdot16}\cdot 6 = 0.0117 \qquad [\mathrm{mm}].
% \end{equation}



% % ---   ---   --- %

% \subsection{Tablas}

% Ejemplo de tabla de grandes dimensiones e introducir una página apaisada. También se muestra como poner en negrita letras griegas, que a veces dan problemas.

% \begin{landscape}
%     \vspace*{\fill}
%     \input{Tables/table_example}
%     \vspace*{\fill}
%     \clearpage
% \end{landscape}


% % ---   ---   --- %

% \subsection{Código}


% \begin{lstlisting}
% M303 H0 S60 ; auto tune heater 0, default PWM (100%), 60C target
% M303        ; report the auto-tune status or last resulM303 ; report the auto-tune status or last result
% M500        ; save parameters
% \end{lstlisting}

% \lstinputlisting{Code/function.m}



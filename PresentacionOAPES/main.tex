\documentclass{cubeamer}
\usepackage{lmodern}
\usepackage{graphicx}

\title{Planck Thermal Control System.}
%\subtitle{Conference Name}
\author[Inés Arauzo Andres]{Inés Arauzo Andrés}
\date{\today} % or whatever the date you are presenting in is
\institute[UPM]{Heat transfer and thermal control \\

\footnotesize{MUSE  IDR-UPM-ETSIAE}}


% \copyrightnotice{Published by the American Institute of Aeronautics and Astronautics, Inc., with permission}

\begin{document}

\maketitle

\cutoc

\section{Introduction}

\begin{frame}{What is Planck?}
    \begin{minipage}{0.65\textwidth}
    \begin{itemize}
        \item Planck is the third generation space mission to measure the anisotropy of the cosmic microwave background (CMB). 
        \item It orbits the second Lagrange point (L2)
        \item It observes the sky in nine frequency bands covering from 27GHz to 1THz.
        \item It has two main instruments:
        \begin{enumerate}
            \item The Low Frequency Instrument (LFI).
            \item The High Frequency Instrument (HFI).
        \end{enumerate}
    \end{itemize}
    \end{minipage}
    \begin{minipage}{0.3\textwidth}
    \includegraphics[width=0.9\linewidth]{Figures/Planck1.jpg} \\
    \includegraphics[width=0.9\linewidth]{Figures/Planck2.jpg}
    \end{minipage}
    
\end{frame}

\section{Mission requirements and constrains}

\begin{frame}{Thermal requirements}
\begin{table}[H]
\begin{center}
\caption{Operating temperatures of the main equipments.}
\label{tab:ej2}
\begin{tabular}{c c c c c}
\toprule
\textbf{Instrument} & \textbf{Solar Panel} &  \textbf{SVM} & \textbf{LFI} & \textbf{HFI} \\ \midrule
\textbf{ $\boldsymbol{T}$ [K]} & 385 & 275 & 20 & 0.1 \\
\bottomrule
\end{tabular}
\end{center}
\end{table}

   \begin{itemize}
   \item HFI environment temperature < 2K.
   \item LFI environment temperature < 60K.
   \item 0.5 W heatlift in the LFI.
   \end{itemize}

\end{frame}

\begin{frame}{Other requirements}

   
\begin{itemize}
   \item No deployables.
   \item No optical elements such as windows with warm edges between the feed horns and telescope 
   \item 1.5 yr minimum total lifetime.
   \item A spinning spacecraft.
   \end{itemize}   
\end{frame}

\section{Precedents}
\begin{frame}{Background}

    \begin{minipage}{0.4\textwidth}
    \includegraphics[width=0.9\linewidth]{Figures/planck3.jpg} \\
    \includegraphics[width=0.9\linewidth]{Figures/herschel.jpg}
    \end{minipage}
    \begin{minipage}{0.55\textwidth}

   
         \begin{itemize}
         \item COBE ($T=2.725 \text{K}$): used a 650 l He cryostat to cool the IR spectro-photometer to 2 K .
         \item WMAP ($T=90 \text{K}$): used passive cooling. A lot simpler, but unable to reach sub-Kelvin operating temperatures.
         \item Herschel ($T=1.65 \text{K}$):  used 2 He cryostats of 2300 l each to cool the HIFI.

         \end{itemize}
    \end{minipage}
\end{frame}




\section{Thermal control instruments}
\begin{frame}{Thermal control instruments}

    
    \begin{minipage}{0.5\textwidth}

   \begin{enumerate}
       \item Passive Elements
         \begin{itemize}
         \item Telescopic Baffle: provided radiative shield.
         \item 3 V-groove radiators.
         \end{itemize}
         \item Active Elements
         \begin{itemize}
         \item Hydrogen sorption cooler.
         \item Helium Joule Thompson expansion cooler.
         \item He-He dilution cooler.

         \end{itemize}
    \end{enumerate}
    \end{minipage}
    \begin{minipage}{0.45\textwidth}
    \includegraphics[width=0.9\linewidth]{Figures/cut_T_view.png} 
    \end{minipage}
\end{frame}

\begin{frame}{Passive elements}

    \begin{minipage}{0.3\textwidth}
    \includegraphics[width=0.9\linewidth]{Figures/vgroves.jpg} \\
    \includegraphics[width=0.9\linewidth]{Figures/V-groove_radiators.jpg}
    \end{minipage}
    \begin{minipage}{0.65\textwidth}

   
         \begin{itemize}
         \item Telescopic Baffle: provided radiative shield.
         \item 3 V-groove radiators: provided  thermal isolation and radiative cooling
         \end{itemize}
         Allows to reach the precooling temperature $\leq60$K.
    \end{minipage}
\end{frame}



\begin{frame}{Active elements}
\begin{minipage}{0.3\textwidth}
    \includegraphics[width=0.9\linewidth]{Figures/H_sorption_cooler.png} 
    \small \centering  H-Sorption cooler
\end{minipage}
\begin{minipage}{0.3\textwidth}
    \includegraphics[width=0.9\linewidth]{Figures/He_JT_cooler.png} 
    \small \centering  He-JT Expansion cooler
\end{minipage}
\begin{minipage}{0.3\textwidth}
    \includegraphics[width=0.9\linewidth]{Figures/He_He_dilution.png} 
    \small \centering  He-He dilution cooler
\end{minipage}
\end{frame}




\begin{frame}{Active elements: H-Sorption Cooler}
\centering

    \includegraphics[width=0.7\linewidth]{Figures/Hsorption_cooler_diag.png} \\
    Functioning of an sorption cooler
\end{frame}




\begin{frame}{Active elements: He-JT cooler}
\begin{minipage}{0.4\textwidth}
    \includegraphics[width=0.9\linewidth]{Figures/JTHE.png} 
\end{minipage}
\begin{minipage}{0.5\textwidth}
\begin{itemize}
    \item JT He cooler works by expanding helium gas through a small orifice, resulting in a temperature drop. 
    \item Cold He recirculates through a heat exchanger to cool down a target object.
    \item He is compressed back to its original pressure to start the cooling cycle again.
\end{itemize}
\end{minipage}
\end{frame}




\begin{frame}{Active elements: He-He Dilution cooler}

    \begin{minipage}{0.3\textwidth}
            \includegraphics[width=0.8\linewidth]{Figures/He_He_dilution_diag.png} 

    \end{minipage}
    \begin{minipage}{0.65\textwidth}
    \begin{itemize}
        \item Cooling medium is a mixture of Helium-3 and Helium-4 gradually cooled down to temperatures approaching 0 K. 
        \item Uses the superfluid helium-4 to carrying heat away from the target object, as it flows without any resistance.
        \item Can achieve extremely low temperatures, down to a few millikelvin.
    \end{itemize}
    \end{minipage}

\end{frame}
\begin{frame}{Complete system}
\begin{minipage}{0.45\textwidth}
    \includegraphics[width=0.9\linewidth]{Figures/complete.png} 
\end{minipage}
\begin{minipage}{0.45\textwidth}
    \includegraphics[width=0.9\linewidth]{Figures/Diag_complete.png} 
\end{minipage}
\end{frame}
\section{Future missions}
\begin{frame}{Active Elements}
\begin{minipage}{0.3\textwidth}
    \includegraphics[width=0.9\linewidth]{Figures/Core.jpg} 
    \small \centering  COrE+ \\ Cosmic Origins Explorer, ESA
\end{minipage}
\begin{minipage}{0.3\textwidth}
    \includegraphics[width=0.9\linewidth]{Figures/litebird.png} 
    \small \centering  LiteBird \\Primordial B modes mission, JAXA
\end{minipage}
\begin{minipage}{0.3\textwidth}
    \includegraphics[width=0.9\linewidth]{Figures/ìxie.png} 
    \small \centering  Pixie Absolute Spectrophotometer, NASA
\end{minipage}
\end{frame}

\section{References}
\begin{frame}{References}
\begin{itemize}
\small
    \item López-Caraballo, C. H., Suárez, et al. (2011). On the stability and performance of the Planck satellite’s radiometers. Astronomy \& Astrophysics, 536, A12. doi: 10.1051/0004-6361/201116486.
    \item Leroy, C., Arondel, A., Bernard, J., ..., \& Planck-HFI Instrument Team. (2011). Performances of the Planck-HFI cryogenic thermal control system. Astronomy \& Astrophysics, 536, A22. doi: 10.1051/0004-6361/201116486
    \item Bhandari, P., Moore, B., Bolton, D., \& Aboobaker, A. (n.d.). Design and Analysis of V-Groove Passive Cryogenic Radiators for Space-borne Telescopes \& Instruments. Jet Propulsion Laboratory, California Institute of Technology, Pasadena, CA 91109.
    \item Ranajoy Banerji. Optimisation of a 4th generation CMB space mission. Cosmology and Extra-Galactic Astrophysics. Université Paris Diderot .
\end{itemize}
\end{frame}

 

\begin{frame}[standout]
    \Huge\textsc{Thank you for attending}
    
    \vfill
    
    \LARGE\textsc{Questions?}
\end{frame}






\end{document}

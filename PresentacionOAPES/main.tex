<<<<<<< HEAD
\2documentclass{cubeamer}
\usepackage{lmodern}
=======
\documentclass{cubeamer}
>>>>>>> 189d2cc461b3a7d19082cbd6d9c79f5b68fd304e
\usepackage{graphicx}
\usepackage[inkscapeformat=png]{svg}

\title{Optimization of thermal sensor placement.}
%\subtitle{Conference Name}
\author[Inés Arauzo Andres]{Inés Arauzo Andrés}
\date{\today} % or whatever the date you are presenting in is
\institute[UPM]{OAPES Project \\

\footnotesize{ IDR-UPM-ETSIAE}}


% \copyrightnotice{Published by the American Institute of Aeronautics and Astronautics, Inc., with permission}

\begin{document}
\maketitle
\cutoc

\section{Introduction}

<<<<<<< HEAD
\begin{frame}{State of the art}
    \begin{minipage}{0.65\textwidth}
    \begin{itemize}
        \item Thermal mathematical model of the UPMSat 3 in ESATAN. 
        \item $\approx$ 1150 nodes
        \item Two steady cases defined by the mission requirements:
        \begin{enumerate}
            \item Steady hot with all the payloads active (PL1 Case).
            \item Steady Cold, latency case, minimum power consumption.
        \end{enumerate}
    \end{itemize}
=======
\begin{frame}{UPMSat 3 Thermal Model}
\begin{minipage}{0.4\textwidth}
      Thermal mathematical model parametrized in ESATAN with 2 steady cases:
    \begin{center}
    \begin{tabular}{cc}
    \toprule
    \textbf{Cold Case }[$W$] & \textbf{Hot Case } [$W$]\\ \midrule
    $\approx$ 1 & $\approx$ 7\\
    \bottomrule
    \end{tabular}
    \end{center}  
\end{minipage}
        \begin{minipage}{0.5\textwidth}
    \begin{flushright}
        \includegraphics[width=0.9\linewidth]{Figures/UPMSAT3-esatan.png}
    \end{flushright} 
>>>>>>> 189d2cc461b3a7d19082cbd6d9c79f5b68fd304e
    \end{minipage}
\end{frame}

\begin{frame}{Objectives}
\textbf{Develop method to optimize thermocouple positioning in order to recover the maximum information possible}
    \begin{itemize}
        \item Validate ESATAN model with Pycanha.
        \item Develop TVAC test cases in ESATAN.
        \item Determine influential parameters in TVAC and flight cases.
        \item Optimize sensor placement for maximum information.
        \item Perturb and correlate the model.
    \end{itemize}
\end{frame}


<<<<<<< HEAD
\begin{frame}{ESATAN vs Pycanha}


   \begin{itemize}
   \item HFI environment temperature < 2K.
   \item LFI environment temperature < 60K.
   \item 0.5 W heatlift in the LFI.
   \end{itemize}
=======
\section{ESATAN Model Validation}
>>>>>>> 189d2cc461b3a7d19082cbd6d9c79f5b68fd304e

\begin{frame}{TVAC Cases vs Pycanha}
TVAC cases mirror  hot and cold flight scenarios in BC's. \\
TVAC radiative case: enclosure 0 and 25 ºC for cold and hot scenarios respectively.
    \begin{center}
    \begin{tabular}{cccc}
    \toprule
    \textbf{Cold TVAC} & \textbf{Hot TVAC} & \textbf{Cold Flight} & \textbf{Hot Flight} \\
    \midrule
    $6\cdot 10^{-3}$ & $4\cdot 10^{-2}$ & $2\cdot 10^{-2}$ & $2\cdot 10^{-2}$ \\
    \bottomrule
    \end{tabular}
    \end{center}
\end{frame}

\section{Parameter Influence}

\begin{frame}{Parameter Influence Diagram}
    \begin{center}
    \includesvg[width=0.7\linewidth]{Figures/diagrama.svg}
    \end{center}
\end{frame}
\section{Influence Matrix}
\begin{frame}{ Influence Matrix Definition}
\begin{itemize}
    \item $M = $ Jacobian or sensitivity matrix (forward differences).
    \item $\delta X = $ parameter allowed variation (preeliminary defined as $0.1X$).
\end{itemize}
\begin{center}
\begin{equation}
\mathbf{I}_{\mathbf{X}}=\left[\begin{array}{cccc}
\frac{\partial T_1}{\partial X_1} \delta X_1 & \frac{\partial T_1}{\partial X_2} \delta X_2 & \ldots & \frac{\partial T_1}{\partial X_{N_P}} \delta X_{N_P} \\
\ldots & \ldots & & \ldots \\
\frac{\partial T_{N_N}}{\partial X_1} \delta X_1 & \frac{\partial T_{N_N}}{\partial X_2} \delta X_2 & \ldots & \frac{\partial T_{N_N}}{\partial X_{N_P}} \delta X_{N_P}
\end{array}\right]=\mathbf{M} \delta \boldsymbol{X}
\end{equation}

\end{center}

\end{frame}

\begin{frame}{Influence Matrix (TVAC)}
\begin{center}
    \begin{minipage}{0.475\textwidth}
    \includegraphics[width=1\linewidth]{Figures/TVAC/infmatCC.png}
\end{minipage}
\begin{minipage}{0.475\textwidth}
    \includegraphics[width=1\linewidth]{Figures/TVAC/infmatHC.png}
\end{minipage}
\end{center}

\end{frame}

\begin{frame}{Influence Matrix (FLIGHT)}
\begin{center}
    \begin{minipage}{0.475\textwidth}
    \includegraphics[width=1\linewidth]{Figures/Flight/infmatCC-F.png}
\end{minipage}
\begin{minipage}{0.475\textwidth}
    \includegraphics[width=1\linewidth]{Figures/Flight/infmatHC-F.png}
\end{minipage}
\end{center}

\end{frame}

\begin{frame}{Combined Influence Matrix (TVAC)}
\begin{center}

    \includegraphics[width=0.9\linewidth]{Figures/TVAC/infmatTC.png}
\end{center}

\end{frame}

\begin{frame}{Normalized Global Influence Matrix (TVAC)}
\begin{center}
 
    \includegraphics[width=0.9\linewidth]{Figures/TVAC/global-inf-TC.png}

\end{center}

\end{frame}

\begin{frame}{Reduced Global Influence Matrix (TVAC)}
\begin{center}

    \includegraphics[width=0.7\linewidth]{Figures/TVAC/parameters-imp-tvac2.png}
\end{center}

\end{frame}

\begin{frame}{Reduced Global Influence Matrix (FLIGHT)}
\begin{center}
    \includegraphics[width=0.7\linewidth]{Figures/Flight/indep-iinfglobalboth.png}
\end{center}
\end{frame}

\section{Possible scenarios}
\begin{frame}{Depending on parameter importance}
    \begin{itemize}
        \item Flight case parameters $\subseteq$ Test Case parameters: Test case parameters will be able to model the flight scenario accurately.
        \\
        \item  Flight case parameters $\not \subseteq$ Test Case parameters: More and/or different test cases must be defined.
    \end{itemize}
\end{frame}
\begin{frame}{Depending on number of sensors}
    \begin{itemize}
        \item Number of sensors $>$ Number of parameters $\longrightarrow$ solve with least squares to minimize the error (error $\neq$ 0)
        \item Number of sensors $=$ Number of parameters $\longrightarrow$ place sensor in most relevant nodes and solve with least squares to minimize the error (error $\approx$ 0)
        \item Number of sensors $=$ Number of parameters:
        \begin{enumerate}
            \item Use an SVD to determine which nodes give more information according to the eigenvalues.
            \item Develop more test cases
        \end{enumerate}
    \end{itemize}
\end{frame}
\section{Node selection}
\begin{frame}{Sensible nodes}
\begin{center}

    \begin{minipage}{0.45\textwidth}
    \includegraphics[width=0.9\linewidth]{Figures/nodos1.png}
\end{minipage}
\begin{minipage}{0.5\textwidth}
14 nodes:
    \begin{itemize}
        \item 4 on the trays
        \item 2 on the solar panels
        \item 2 on the radiator frame
        \item 2 on UC3M butterfly 
        \item 1 on Shrinkwrap
        \item 1 on OBC
        \item 1 on Battery module
        \item 1 on UC3M Ebox
    \end{itemize}
\end{minipage}
\end{center}

\end{frame}

\section{Future work}

\begin{frame}{Next Steps}
    \begin{itemize}
        \item Refine influence matrices with realistic parameter variations ($\delta X$) and/or giving weights.
        \item With new influence, refine the selected parameters to see if test cases still contain flight parameters.
        \item Revisit over-determined and under-determined cases.
        \item Begin test campaign.
    \end{itemize}
\end{frame}

\section{References}
\begin{frame}{References}
\begin{itemize}
\small
    \item A. Avila, “JPL Thermal Design Modeling Philosophy and NASA-STD-7009
Standard for Models and Simulations - A Case Study,” National Aeronautics
and Space Administration, pp. 1–13, 2011.
    \item M. Molina and C. Clemente, “Thermal Model Automatic Reduction: Algorithm
and Validation Techniques,” in 19th European Workshop on Thermal and ECLS
Software, no. 36, 2005
    \item Mercer, J.F., Aglietti, G. and Remedia, M. (no date) Containment of the spacecraft finite element model correlation process. thesis. 
    \item M. Bernard, T. Basset, F. Brunetti, and J. Etchells, “Thermal Model Reduc-
tion Tool,” in 24th European Workshop on Thermal and ECLS Software, no.
November, 2010, pp. 1–21..
\end{itemize}
\end{frame}

 

\begin{frame}[standout]
    \Huge\textsc{Thank you for attending}
    
    \vfill
    
    \LARGE\textsc{Questions?}
\end{frame}






\end{document}
